\documentclass[10pt,aspectratio=169]{beamer}
\usepackage[utf8]{inputenc}
\usepackage{listings}
\usepackage{hyperref}
\usepackage{xcolor}
\usepackage{ragged2e}
\usepackage{graphicx}

\usetheme{Dresden}
%\usetheme{Bergen}

\title{Programming Techniques 2025-2026}
\author{Hannu Parviainen}
\institute{Universidad de la Laguna}
\date{\today}

\lstset{
  language=fortran,
  basicstyle=\ttfamily\scriptsize,
  backgroundcolor=\color{white},
  keywordstyle=\color{blue},
  commentstyle=\color{gray},
  stringstyle=\color{red},
  showstringspaces=false,
  breaklines=true,
  frame=lines
}

\subtitle{Lecture 1: Introduction and Version Control}

\begin{document}

\begin{frame}
  \titlepage
\end{frame}

\section{Introduction}

\begin{frame}{Programming Techniques 2025--2026}
  \begin{itemize}
    \item Lecturer: Hannu Parviainen (ULL, IAC)
    \item Language: English
    \item Contact: \texttt{hparviai@ull.edu.es}
    \item Schedule: Mondays and Wednesdays 13:00--15:00
    \item Dates: 10.9.2025 -- 2X.11.2025
    \item Location: CCA
    \item Structure: theory + exercises
    \item Reading: Fortan book, Pro Git, old lecture notes
  \end{itemize}
\end{frame}

\begin{frame}{Programming Techniques 2025--2026}
	\begin{block}{Assessment options:}
		\begin{enumerate}
			\item Two equally weighted exercises, no final exam
			\begin{itemize}
				\item Both exercises can be done over the period of several weeks
				\item GitHub is used to follow and return the exercises, as well as to discuss and share ideas
				\item Scoring is detailed separately for each exercise. However, the main criteria are that the code works, is readable and is well-documented. Interaction and involvement in GitHub is also important.
			\end{itemize}
			\vspace{0.25cm}
			\item Final exam: writing a fully working parallelised Fortran program in 4 h (paper + pen)
			\begin{itemize}
				\item Nobody has ever chosen this
			\end{itemize}
		\end{enumerate}
	\end{block}
\end{frame}


\begin{frame}{Course Topics}
  \begin{itemize}
  	\item Programming techniques in astronomy and astrophysics
    \item Modern Fortran
      \begin{itemize}
        \item Data types
        \item Flow control structures
        \item Modules, subroutines and functions
        \item Dynamic memory management (allocatable arrays, pointers)
        \item Data structures: linked lists and trees
      \end{itemize}
    \item Debugging
    \item Parallelisation: OpenMP
    \item Distributed computing: MPI
    \item Version control: Git + GitHub
    \item Makefiles
  \end{itemize}
\end{frame}


\begin{frame}{Not Course Topics}
	\begin{block}{IDEs}
	\begin{itemize}
		\item Course teaches programming, not how to use a specific Integrated Development Environment
		\item The code can be written using a text editor and compiled using command-line commands
		\item \ldots however, a good IDE can make the course way less painful
		\item Good options: \href{https://code.visualstudio.com}{VSCode}, \href{https://www.codeblocks.org/}{Code::Blocks}, Emacs (old-school)
		\item Check the list from \href{https://fortran-lang.org/learn/os_setup/ides/}{fortran-lang.org}
	\end{itemize}
	\end{block}
\end{frame}


\begin{frame}{Lecturer: Hannu Parviainen}
  \begin{itemize}
    \item Research: exoplanets, Bayesian statistics, scientific computing, radiative transfer
    \item BASIC, PASCAL, C, and C++ 1991--2004
    \item Fortran as main language 2004--2017
    \item Now: Python + Numba + OpenCL + JAX
    \item Offices at IAC and ULL — feel free to pass by
  \end{itemize}
\end{frame}

\begin{frame}{Basic Workflow}
	\begin{enumerate}
		\item Fork the official course repository on GitHub into your own account
		\item Clone your forked repository to your local computer
		\item Work on exercises in your own computer
		\item Push changes back to your fork on GitHub
		\item Open pull requests to merge your work to the main course repository
	\end{enumerate}
\end{frame}

\section{Fortran}

\begin{frame}
	\centering
	\Huge \textbf{Introduction to Fortran}
\end{frame}

% Fortran overview
\begin{frame}{Fortran}
	\begin{itemize}
		\item Compiled programming language (vs. interpreted like Python)
		\item Developed in the 1950s for scientific computing
		\item Very good for numerically heavy problems in physics
		\item Not really good for much else… but works well together with Python!
		\item Major versions: FORTRAN 77, Fortran 90/95, 2003, 2008, 2018, 2023
	\end{itemize}
\end{frame}



\begin{frame}{Compiled vs Interpreted Languages}
	\begin{columns}[T]
		\column{0.48\textwidth}
		\textbf{Compiled (Fortran, C, C++)}
		\begin{itemize}
			\item Source $\to$ machine code via compiler
			\item Produces executable before running
			\item Optimised for target CPU
			\item Requires a compiler
		\end{itemize}
		
		\column{0.48\textwidth}
		\textbf{Interpreted (Bash, Python, R, Matlab)}
		\begin{itemize}
			\item Source executed line by line
			\item No separate compilation step
			\item Requires interpreter to run
		\end{itemize}
	\end{columns}
\end{frame}


\begin{frame}{Performance}
	\begin{columns}[T]
		\column{0.48\textwidth}
		\textbf{Compiled}
		\begin{itemize}
			\item Very fast execution
			\item Close to native machine code
			\item Often 10--100$\times$ faster than Python \\ (but this can be improved using Numba or jAX)
		\end{itemize}
		
		\column{0.48\textwidth}
		\textbf{Interpreted}
		\begin{itemize}
			\item Slower due to interpreter overhead
			\item Performance depends on libraries
		\end{itemize}
	\end{columns}
\end{frame}


\begin{frame}{Development Speed}
	\begin{columns}[T]
		\column{0.48\textwidth}
		\textbf{Compiled}
		\begin{itemize}
			\item Slower cycle: edit $\to$ compile $\to$ run
			\item More boilerplate code
		\end{itemize}
		
		\column{0.48\textwidth}
		\textbf{Interpreted}
		\begin{itemize}
			\item Rapid iteration: edit $\to$ run
			\item Shorter programs, easier syntax
		\end{itemize}
	\end{columns}
\end{frame}

\begin{frame}{Portability}
	\begin{columns}[T]
		\column{0.48\textwidth}
		\textbf{Compiled}
		\begin{itemize}
			\item The code needs to be compiled for each OS and CPU architecture separately
		\end{itemize}
		
		\column{0.48\textwidth}
		\textbf{Interpreted}
		\begin{itemize}
			\item The code can be run with any setup (if the interpreter exists)
		\end{itemize}
	\end{columns}
\end{frame}


\begin{frame}{Error Handling}
	\begin{columns}[T]
		\column{0.48\textwidth}
		\textbf{Compiled}
		\begin{itemize}
			\item Errors caught at compile time
			\item Type mismatches, missing variables flagged early
		\end{itemize}
		
		\column{0.48\textwidth}
		\textbf{Interpreted}
		\begin{itemize}
			\item Errors appear at runtime
			\item Flexible, but can fail late
		\end{itemize}
	\end{columns}
\end{frame}


\begin{frame}{Typical Use Cases}
	\begin{columns}[T]
		\column{0.48\textwidth}
		\textbf{Compiled}
		\begin{itemize}
			\item Heavy numerical simulations
			\item High-performance computing
			\item Long-running production code
		\end{itemize}
		
		\column{0.48\textwidth}
		\textbf{Interpreted}
		\begin{itemize}
			\item Data analysis and exploration
			\item Rapid prototyping
			\item Teaching, scripting, glue code
		\end{itemize}
	\end{columns}
\end{frame}


\begin{frame}[fragile]{Fortran: First Steps}
		\begin{columns}[T]
		\column{0.48\textwidth}
		\begin{itemize}
		\item Program code inside a \texttt{program ... end program} block
		\item Always use \texttt{implicit none}
		\item Simple printing: \texttt{print *, ...}
		\end{itemize}
		\column{0.48\textwidth}
		\begin{lstlisting}[language=Fortran]
		program hello
		implicit none
		print *, "Hello world!"
		end program hello
		\end{lstlisting}
	\end{columns}
\end{frame}

\begin{frame}[fragile]{Compiling Fortran}
	\begin{itemize}
		\item Source code file $\to$ executable program
		\item Free compilers: GNU, Intel, AMD
		\item Vendor compilers optimise for CPU, but GNU is usually enough
	\end{itemize}
	\begin{lstlisting}[language=Fortran]
	gfortran -o hello hello.f90
	./hello
	\end{lstlisting}
\end{frame}

\begin{frame}{Fortran Source Files}
	\begin{itemize}
		\item Use suffix \texttt{.f90} for modern Fortran
		\item Avoid version-specific suffixes (\texttt{.f03}, \texttt{.f08}, …)
	\end{itemize}
\end{frame}


\section{Version Control}
\begin{frame}
	\centering
	\Huge \textbf{Version Control}
\end{frame}

\begin{frame}{Version Control: What It Is and Why We Need It}
	\begin{itemize}
		\item Tracks changes in code, documents, and data
		\item Allows you to:
		\begin{itemize}
			\item Revert to earlier versions
			\item Compare changes over time
			\item Work on features without breaking main code
		\end{itemize}
		\item Essential for collaboration
		\item Prevents the "final\_v2\_really\_final.f90" problem
	\end{itemize}
\end{frame}

\begin{frame}{Version Control Concepts}
	\begin{itemize}
		\item \textbf{Repository}: central place for files + history
		\item \textbf{Commit}: snapshot of project at a given time
		\item \textbf{Branch}: parallel line of development
		\item \textbf{Merge}: combine changes from different branches
		\item \textbf{Remote}: shared repository for collaboration
	\end{itemize}
	\vspace{0.5em}
	Popular systems: Git, Subversion (SVN), Mercurial  
\end{frame}


\section{Version Control with Git}
\begin{frame}
	%\frametitle{What is Git?}
	\begin{block}{What is Git?}
		\textbf{Distributed Version Control System (VCS)} 
		\begin{itemize}
			\item Tracks changes in the source code and helps coordinate work among programmers.
			\item Designed to handle everything from small to very large projects quickly and efficiently. 
			\item Allows multiple developers to work on a project simultaneously without overwriting each other's changes. 
		\end{itemize}
	\end{block}
	
	\begin{block}{Distributed?}
		\justifying
		\begin{itemize}
			\item Each user has a full copy of the repository.
			\item Allows for a more flexible workflow than centralised VCSs.
			\item Branching is fast and does not require a connection to the VCS server.
		\end{itemize}
	\end{block}
	
\end{frame}


\begin{frame}
	\frametitle{Why Use Git?}
	\begin{itemize}
		\item \textbf{Collaboration:} Streamlines teamwork on projects.
		\item \textbf{Backup and Restore:} Safeguards code with version history.
		\item \textbf{Track History:} Allows you to see what changes were made and when.
		\item \textbf{Branching and Merging:} Supports multiple development lines.
	\end{itemize}
\end{frame}

\begin{frame}
	\frametitle{Installing Git}
	\begin{block}{Windows}
		Install via \href{https://git-scm.com/download/win}{Git for Windows}.
	\end{block}
	\begin{block}{Mac}
		Install via Homebrew using `brew install git` or install \href{https://developer.apple.com/xcode/}{Xcode}.
	\end{block}
	\begin{block}{Linux}
		Install via package manager using `sudo apt-get install git` or equivalent.
	\end{block}
\end{frame}

\begin{frame}[fragile]
	\frametitle{Configuring Git}
	Set your username and email for Git commits:
	\begin{lstlisting}[language=bash]
	$ git config --global user.name "Your Name"
	$ git config --global user.email "you@example.com"
	\end{lstlisting}
	\begin{itemize}
		\item These settings are used to attribute commits to you.
		\item The `--global` flag applies settings for all repositories.
	\end{itemize}
\end{frame}

%\section{Basic Git Commands}

\begin{frame}[fragile]
	\frametitle{Initializing a Repository}
	Initialize a new Git repository in the current directory:
	\begin{lstlisting}[language=bash]
	$ git init
	\end{lstlisting}
	\begin{itemize}
		\item Creates a new `.git` subdirectory.
		\item Starts tracking versions for your project.
	\end{itemize}
\end{frame}

\begin{frame}[fragile]
	\frametitle{Checking Status}
	Check the status of your working directory and staging area:
	\begin{lstlisting}[language=bash]
	$ git status
	\end{lstlisting}
	\begin{itemize}
		\item Shows tracked and untracked files.
		\item Indicates changes that are staged for commit.
	\end{itemize}
\end{frame}

\begin{frame}
	\frametitle{How Git Tracking Works}
	\begin{itemize}
		\item \textbf{Working Directory:} Your local filesystem where you modify files.
		\item \textbf{Staging Area (Index):} A temporary area where you add changes to prepare for a commit.
		\item \textbf{Repository (History):} The database where commits are stored.
		\item \textbf{Tracking Changes:}
		\begin{itemize}
			\item Git monitors changes in tracked files.
			\item Untracked files are not monitored until added.
		\end{itemize}
		\item \textbf{Lifecycle of a File:}
		\begin{enumerate}
			\item Modify files in the working directory.
			\item Stage changes using `git add`.
			\item Commit changes to the repository with `git commit`.
		\end{enumerate}
	\end{itemize}
\end{frame}

\begin{frame}
	\frametitle{Staging vs. Committing Changes}
	\begin{itemize}
		\item \textbf{Staging Changes (\texttt{git add}):}
		\begin{itemize}
			\item Prepares selected changes for the next commit.
			\item Allows you to review and group changes.
		\end{itemize}
		\item \textbf{Committing Changes (\texttt{git commit}):}
		\begin{itemize}
			\item Records the staged changes into the repository history.
			\item Creates a new commit object with a unique ID.
			\item Includes a commit message describing the changes.
		\end{itemize}
		\item \textbf{Key Differences:}
		\begin{itemize}
			\item \textbf{Staging:} Prepares changes, but does not save them to history.
			\item \textbf{Committing:} Saves the staged changes permanently in the repository.
		\end{itemize}
	\end{itemize}
\end{frame}

\begin{frame}[fragile]
	\frametitle{Staging Changes with \texttt{git add}}
	Add files to the staging area:
	\begin{lstlisting}[language=bash]
	$ git add filename
	$ git add .
	\end{lstlisting}
	\begin{itemize}
		\item \texttt{git add filename}: Stages a specific file.
		\item \texttt{git add .}: Stages all changes in the current directory.
		\item \textbf{Staging:} Prepares changes to be included in the next commit.
	\end{itemize}
\end{frame}

\begin{frame}[fragile]
	\frametitle{Committing Changes with \texttt{git commit}}
	Commit the staged changes to the repository:
	\begin{lstlisting}[language=bash]
	$ git commit -m "Commit message"
	\end{lstlisting}
	\begin{itemize}
		\item Records a snapshot of the staging area.
		\item \textbf{Commit Message:} Describes the changes made.
	\end{itemize}
\end{frame}


\begin{frame}
	\frametitle{Understanding Branches}
	\begin{itemize}
		\item \textbf{What is a Branch?}
		\begin{itemize}
			\item A parallel version of the repository.
			\item Allows you to work on different features independently.
		\end{itemize}
		\item \textbf{Why Use Branches?}
		\begin{itemize}
			\item Isolate development work without affecting the main codebase.
			\item Facilitate collaboration by allowing multiple features to be developed simultaneously.
		\end{itemize}
		\item \textbf{Default Branch:} Typically named `main` or `master`.
	\end{itemize}
\end{frame}

\begin{frame}[fragile]
	\frametitle{Creating and Switching Branches}
	Create a new branch and switch to it:
	\begin{lstlisting}[language=bash]
	$ git branch new-feature
	$ git checkout new-feature
	\end{lstlisting}
	Or combine both steps:
	\begin{lstlisting}[language=bash]
	$ git checkout -b new-feature
	\end{lstlisting}
	\begin{itemize}
		\item \texttt{git branch new-feature}: Creates a new branch named `new-feature`.
		\item \texttt{git checkout new-feature}: Switches to the `new-feature` branch.
		\item \texttt{git checkout -b new-feature}: Creates and switches to `new-feature` in one command.
	\end{itemize}
\end{frame}

\begin{frame}[fragile]
	\frametitle{Merging Branches}
	Merge changes from `new-feature` branch into `main`:
	\begin{lstlisting}[language=bash]
	$ git checkout main
	$ git merge new-feature
	\end{lstlisting}
	\begin{itemize}
		\item \texttt{git checkout main}: Switches to the `main` branch.
		\item \texttt{git merge new-feature}: Merges `new-feature` into `main`.
		\item \textbf{Merging:} Combines changes from one branch into another.
	\end{itemize}
\end{frame}

\section{Introduction to GitHub}

\begin{frame}
	\frametitle{What is GitHub?}
	\begin{itemize}
		\item \textbf{Code Hosting Platform:} Hosts Git repositories online.
		\item \textbf{Facilitates Collaboration:} Tools for team communication and coordination.
		\item \textbf{Additional Features:}
		\begin{itemize}
			\item Issue tracking.
			\item Pull requests for code reviews.
			\item Wiki pages and documentation.
		\end{itemize}
	\end{itemize}
\end{frame}

\begin{frame}
	\frametitle{Setting Up GitHub}
	\begin{itemize}
		\item \textbf{Create an Account:} Sign up at \href{https://github.com}{github.com}.
		\item \textbf{Set Up SSH Keys:}
		\begin{itemize}
			\item Generate an SSH key pair on your local machine.
			\item Add the public key to your GitHub account.
		\end{itemize}
		\item \textbf{Configure Git to Use SSH:}
		\begin{itemize}
			\item Ensures secure communication with GitHub.
		\end{itemize}
	\end{itemize}
\end{frame}


\begin{frame}[fragile]
	\frametitle{Connecting a Local Repository to GitHub}
	Add a remote repository and push changes:
	\begin{lstlisting}[language=bash]
	$ git remote add origin git@github.com:username/repo.git
	$ git push -u origin main
	\end{lstlisting}
	\begin{itemize}
		\item \texttt{git remote add origin}: Links your local repo to GitHub.
		\item \texttt{git push -u origin main}: Pushes commits to the `main` branch on GitHub.
		\item \textbf{The \texttt{-u} Flag:} Sets `origin main` as the default upstream branch.
	\end{itemize}
\end{frame}

\begin{frame}
	\frametitle{Collaborating on GitHub}
	\begin{itemize}
		\item \textbf{Forking Repositories:} Create a personal copy of someone else's project.
		\item \textbf{Cloning Repositories:} Download a repository to your local machine.
		\item \textbf{Pull Requests:} Propose changes to a repository.
		\item \textbf{Code Reviews:} Collaboratively review code changes before merging.
		\item \textbf{Issue Tracking:} Report bugs or request features.
	\end{itemize}
\end{frame}

\section{Best Practices and Conclusion}

\begin{frame}
	\frametitle{Best Practices}
	\begin{itemize}
		\item \textbf{Write Meaningful Commit Messages:} Clearly describe what changes were made.
		\item \textbf{Keep Commits Small and Focused:} Easier to review and understand.
		\item \textbf{Regularly Pull Updates:} Keep your local repository up-to-date with the remote.
		\item \textbf{Use Branches for New Features:} Isolate development work.
		\item \textbf{Avoid Committing Sensitive Information:} Don't commit passwords or API keys.
	\end{itemize}
\end{frame}

\section{Exercises}

\begin{frame}{Preliminaries}
	\begin{columns}[T]
		\column{0.48\textwidth}
		\begin{block}{Setup a Fortran Compiler}
			\begin{itemize}
				\item Optional: set up a Conda environment
				\item Install gfortran
				\item Optional: install VSCode or another IDE
			\end{itemize}
		\end{block}		
		\column{0.48\textwidth}
		\begin{block}{Setup Git \& GitHub}
			\begin{itemize}
				\item Install Git (if not installed)
				\item Set up a GitHub account
				\item Set up GitHub identification
			\end{itemize}
		\end{block}		
	\end{columns}
\end{frame}


\begin{frame}[fragile]{Preliminaries}
	\begin{block}{Setup Course}
	\begin{itemize}
	 \item On GitHub: click "Fork" on the course repository page
	 \item On your computer: clone your fork
		\begin{lstlisting}[language=bash]
		$ git clone git@github.com:your-username/course-repo.git
		$ cd course-repo
		\end{lstlisting}
	\item Verify your remote
		\begin{lstlisting}[language=bash]
		$ git remote -v
		\end{lstlisting}
	\end{itemize}
	\end{block}
\end{frame}

%\section{Exercises}
%% Exercises
%\begin{frame}{Exercises}
%  \begin{enumerate}
%    \item Modify the “Hello world” program to print it $n$ times, 
%          where $n$ is read from the terminal.
%    \item Write a program that prints factorials $n!$ from 0 to $n$.
%    \item Write a program that prints $r!$ where $r$ is a positive 
%          real number read from the terminal.
%  \end{enumerate}
%\end{frame}

\end{document}
